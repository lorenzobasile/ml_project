\documentclass{article}
\usepackage[utf8]{inputenc}
\usepackage{amsmath,amssymb}
\usepackage{paralist}
\usepackage{color}
\usepackage[detect-weight=true, binary-units=true]{siunitx}
\usepackage{pgfplots}
\usepackage{authblk}
\usepackage{multirow}
\usepackage{subfig}
\usepackage{multicol}
\usepackage{booktabs}
\usepackage{url}
\def\UrlBreaks{\do\/\do-}
\usepackage{breakurl}
\usepackage[breaklinks]{hyperref}
\usepackage[font=small]{caption}

\title{Introduction to Machine Learning project:\\ Leaf identification}
\author[1]{Lorenzo Basile}
\author[2]{Roberto Corti}
\author[3]{Arianna Tasciotti}
\affil[1,2,3]{
    problem statement,
    solution design,
    solution development,
    writing
}


\date{Course of AA 2019-2020}



\begin{document}

\maketitle



\section{Problem statement}
Leaf identification is the process of matching an unknown observed leaf to its proper scientific name. The development of computer technologies has increased the interest inside this field on automatic recognition systems in order to have an inexpensive and fast tool able to classify leafs.  

In this project our aim is to implement several supervised machine learning techniques that could develop a leaf classifier and propose the most effective one according to some performance indexes. \\
The input of this problem is given by 14 morphological and textural features of a picture of the sample leaf and the output is a number associated to its relative species. The classifiers that we present have been trained on a data-set provided by (CITAZIONE Development of a System for Automatic Plant Species Recognition, Silva) that comprises 40 different plant species that has already been pre-processed.


\section{Assessment and 
 performance indexes}



\section{Proposed solution}



\section{Experimental evaluation}

\subsection{Data}

\subsection{Procedure}

\subsection{Results and discussion}



\newpage
\bibliographystyle{plain}
\bibliography{report}


\end{document}
