\documentclass{article}
\usepackage[utf8]{inputenc}
\usepackage{amsmath,amssymb}
\usepackage{paralist}
\usepackage{color}
\usepackage[detect-weight=true, binary-units=true]{siunitx}
\usepackage{pgfplots}
\usepackage{authblk}
\usepackage{multirow}
\usepackage{subfig}
\usepackage{multicol}
\usepackage{changepage}
\usepackage{booktabs}
\usepackage{pbox}
\usepackage{url}
\def\UrlBreaks{\do\/\do-}
\usepackage{breakurl}
\usepackage[breaklinks]{hyperref}
\usepackage[font=small]{caption}
\usepackage{multicol}

\title{Introduction to Machine Learning project:\\ Leaf identification}
\author[1]{Lorenzo Basile}
\author[2]{Roberto Corti}
\author[3]{Arianna Tasciotti}
\affil[1,2,3]{
    problem statement,
    solution design,
    solution development,
    writing
}


\date{Course of AA 2019-2020}



\begin{document}

\maketitle



\section{Problem statement}
Leaf identification is the process of matching an unknown observed leaf to its proper scientific name. Modern automatic recognition systems can provide the experts of this field with a fast and efficient classification tool. 

In this project our aim is to use several supervised machine learning techniques to develop a leaf classifier. \\
The input of this problem is given by 14 morphological and textural features of a picture of a leaf and the output is a number associated to its species. The classifiers that we present have been trained on an already preprocessed dataset provided by \cite{silva} that comprises 40 different plant species.

\section{Assessment and performance indexes}\label{2}
In order to assess how well the classifier would work if applied to unseen data, we used a $5$-fold cross validation (CV) making sure that each class is represented in each fold. We trained different classification models that have been compared using as performance indexes the Accuracy and, for each class, the False Positive Rate (FPR) and the False Negative Rate (FNR).

\section{Proposed solution}\label{3}
The leaf identification problem is a multiclass classification problem with numerical predictors. We focused on two classes of methods: tree based methods and support vector machines (SVM).
Regarding tree based methods, we relied on a simple decision tree and then tried to improve its performance by using tree aggregation methods, such as random forest. This classifier aggregates independent views by making use of multiple decision trees. 
\\SVM are another very powerful machine learning tool, natively built for binary classification: to adapt them to our multiclass task we opted for the $\textit{one versus one}$ approach in which all classes are compared pairwise and the final prediction is the most frequently predicted class. We tried three different kernels to build our SVM, linear, radial and polynomial, and tuned their hyperparameters.
\\Once we trained these classifiers, we compared them by means of the performance indexes presented in Section \ref{2}.

\section{Experimental evaluation}

\subsection{Data}\label{dat}
The dataset used for solving this leaf identification problem is made up of 340 leaf observations of the following 16 features:
\begin{multicols}{4}
	\begin{enumerate}
		
		\footnotesize{\item Class (Species)}
		\footnotesize{\item Specimen Number}
		\footnotesize{\item Eccentricity}
		\footnotesize{\item Aspect Ratio}
		
		\columnbreak
		
		\footnotesize{\item Elongation}
		\footnotesize{\item Solidity}
		\footnotesize {\item Stochastic Convexity}
		\footnotesize{\item Isoperimetric Factor}
		
		\columnbreak
		
		\footnotesize{\item Max. Indentation Depth}
		\footnotesize{\item Lobedness}
		\footnotesize{\item Average\\ Intensity}
		\footnotesize{\item Average Contrast}
		
		\columnbreak
		
		\footnotesize{\item Smoothness}
		\footnotesize{\item Third moment}
		\footnotesize{\item Uniformity}
		\footnotesize{\item Entropy}
	\end{enumerate}
\end{multicols}
\noindent As explained in \cite{silva}, the variable “Class”  assigns each leaf of the dataset to one of $40$ different species encoded as a number. However, we noticed that  of the 40 Species found in the original dataset only 30 (the ones whose leaves are defined as $\textit{simple}$ in \cite{silva}) are present.
Thus, our models will not be able to classify these missing species.
\\ Concerning the other attributes, from 3 to 10 they indicate properties that characterize the shape of the leaf, while from 11 to 16 they refer to texture peculiarities of the image. A full and precise explanation of these variables can be found in \cite{silva}.
\\The attribute "Specimen Number" represents the number of leaf specimens available by species. Hence we thought that this variable was not a useful predictor for our classifiers and we decided to discard it during the preprocessing phase.
\\In order to verify the degree of correlation among the predictors, we produced the correlation matrix and we observed that some variables are highly correlated. We compared the test performance of our classifiers with all predictors and without some redundant ones.

\subsection{Procedure}
In order to tune the hyperparameters and measure the test performance of the methods presented in Section \ref{3}, we performed two nested 5-fold CV. 
In particular, we divided the data into 5 folds ensuring that each class is represented in each fold. Then at each iteration of the outer CV, we assigned one fold to test set and the others to training and we tuned the hyperparameters with an inner 5-fold CV on the corresponding training set. After that, inside the outer CV we assessed the actual performance of our tuned models on the unseen test fold (20\% of the original dataset). Finally, we averaged on the 5 collected values of the Accuracy computed for each model.
\\Regarding decision tree, we tuned its maximum depth, while for random forest we tuned the parameter $m$, which represents the number of features taken into account by each tree, keeping in mind the widely accepted golden standard of fixing it to $\sqrt{p}$ and not allowing it to deviate much from this value.
\\When it comes to SVM, we tuned the regularization parameter for all kernels, the degree and scale factor for polynomial kernel and the scaling parameter $\sigma$ for radial kernel.

\subsection{Results and discussion}
As stated in Section \ref{dat}, we trained the classifiers using all predictors and then without $3$ redundant variables: $\textit{Smoothness}$, $\textit{Average Contrast}$ and $\textit{Maximal}$ $\textit{Indentation Depth}$. 
\\We present the results in terms of Accuracy of the models with all predictors in table \ref{table1}. 
\begin{table}[ht]
 \centering
\begin{tabular}{ |c|c|} \hline
 & Accuracy \\
\hline
Decision Tree & 0.51 \\ 
\hline
 Random Forest & 0.78  \\ 
\hline
 SVM (radial kernel) & 0.76 \\ 
\hline
 SVM (linear kernel) & 0.76 \\ 
\hline
 SVM (polynomial kernel) & 0.75 \\ 
\hline
\end{tabular}
\caption{Accuracy results with all predictors.}\label{table1}
\end{table}

As we expected, decision tree has the worst Accuracy and it does not perform well enough. Tree aggregation methods like random forest produce instead a quite reliable classifier with $0.78$ Accuracy. 
\\Even though SVM takes a completely different approach to the classification problem, not based on decision trees but on separating hyperplanes, it yields similar results in terms of Accuracy. Moreover, we observed that the kernel choice does not impact too much on performance.
\\In table \ref{table2}, we report the Accuracy obtained training the classifiers without the $3$ predictors previously mentioned.
\newpage

\begin{table}[ht]
 \centering
\begin{tabular}{ |c|c|} \hline
 &  Accuracy \\
\hline
Decision Tree & 0.51 \\ 
\hline
 Random Forest & 0.78 \\ 
\hline
 SVM (radial kernel) & 0.74 \\ 
\hline
 SVM (linear kernel) & 0.78  \\ 
\hline
 SVM (polynomial kernel) & 0.74 \\ 
\hline
\end{tabular}
\caption{Accuracy results without $3$ predictors.}\label{table2}
\end{table}

As observed, removing these predictors does not affect the results; however, since for the most accurate models (random forest and SVM with linear kernel) we noticed that the Accuracy increases or stays the same, we think that this design choice makes sense because it reduces the computational cost while ensuring the same Accuracy.
\\Finally, we present in tables \ref{table3} and \ref{table4} the FPR and the FNR of each class for random forest and SVM with linear kernel trained without the $3$ predictors. 

\begin{table}[h]\footnotesize
\begin{minipage}[]{7cm}
 \begin{adjustwidth}{-3cm}{}
\begin{tabular}{|c|c|c|c|c|c|c| } \hline
Species & FPR & FNR & & Species & FPR & FNR \\
\hline
1 & 0.017 & 0.167 & & 22 & 0.008 & 0.133 \\ 
\hline
2 & 0.020 & 0.200 & & 23 & 0.000 & 0.067 \\ 
\hline
 3 & 0.017 & 0.300 & & 24 & 0.020 & 0.333\\ 
\hline
 4 & 0.020 & 0.500 & & 25 & 0.004 & 0.300 \\ 
\hline
 5 & 0.000 & 0.067 & & 26 & 0.023 & 0.700 \\ 
\hline
 6 & 0.004 & 0.000 & & 27 & 0.007 & 0.367 \\ 
\hline
 7 & 0.011 & 0.300 & & 28 & 0.023 & 0.400 \\ 
\hline
 8 & 0.000 & 0.000 & & 29 & 0.003 & 0.067 \\ 
\hline
 9 & 0.024 & 0.233 & & 30 & 0.000 & 0.100  \\ 
\hline
 10 & 0.004 & 0.233 & & 31 &  0.000 & 0.200 \\ 
\hline
 11 & 0.000 & 0.000 & & 32 & 0.015 & 0.367 \\ 
\hline
 12 & 0.012 & 0.367 & & 33 & 0.007 & 0.167\\ 
\hline
 13 & 0.013 & 0.333 & & 34 & 0.008 & 0.000\\ 
\hline
 14 & 0.024 & 0.533 & & 35 & 0.008 & 0.367\\ 
\hline
 15 & 0.000 & 0.000 & & 36 & 0.000 & 0.100\\  
\hline
\end{tabular}
\caption{FPR and FNR for each class obtained \\ with SVM linear kernel classifier.}\label{table3}
\end{adjustwidth}
\end{minipage}
\hfill
\begin{minipage}[]{7cm}
\begin{tabular}{ |c|c|c|c|c|c|c| } \hline
Species & FPR & FNR & & Species & FPR & FNR \\
\hline
1 & 0.010 & 0.267 & & 22 & 0.017 & 0.367 \\ 
\hline
2 & 0.021 & 0.600 & & 23 & 0.000 & 0.100\\ 
\hline
 3 & 0.004 & 0.400 & & 24 & 0.031 & 0.367 \\ 
\hline
 4 & 0.005 & 0.700 & & 25 & 0.000 & 0.100\\ 
\hline
 5 &  0.005 & 0.000 & & 26 & 0.015 & 0.233 \\ 
\hline
 6 & 0.000 & 0.000 & & 27 & 0.012 & 0.233 \\ 
\hline
 7 & 0.015 & 0.200 & & 28 & 0.018 & 0.300\\ 
\hline
 8 & 0.000 & 0.000 & & 29 & 0.003 & 0.067\\ 
\hline
 9 & 0.022 & 0.300 & & 30 & 0.009& 0.100\\ 
\hline
 10 & 0.013 & 0.067 & & 31 &  0.005 & 0.100\\ 
\hline
 11 & 0.003 & 0.000 & & 32 & 0.026 & 0.533\\ 
\hline
 12 & 0.009 & 0.300 & & 33 & 0.010 & 0.533\\ 
\hline
 13 & 0.016 & 0.333 & & 34 & 0.004 & 0.100\\ 
\hline
 14 & 0.011 & 0.267 & & 35 & 0.007 & 0.267\\ 
\hline
 15 & 0.000 & 0.000 & & 36 & 0.000 & 0.100\\ 
\hline
\end{tabular}
\caption{FPR and FNR for each class obtained with random forest classifier.}\label{table4}
\end{minipage}

\end{table}

We can see from FNR values that both classifiers do not perform well on class $4$; this may be due to the fact that there are only $8$ observations for this species. Moreover, we notice that there is some serious imbalance between the two classifiers regarding to FNR of some specific classes. SVM is more sensitive than random forest on classes $2$ and $33$. On the other hand, random forest seems to be more sensitive than SVM on classes $14$ and $26$. %serious imbalance mi suona bene
\\To conclude, random forest and SVM with linear kernel are the best solutions among the tested classifiers. However, there is no general reason to prefer one over the other since Accuracy is the same but the choice of the model may be guided by the necessity to classify some specific classes with high sensitivity.
We could achieve better results by increasing the number of observed leaves in each class, especially for those that have very few observations.


\newpage
\bibliographystyle{plain}
\bibliography{report}


\end{document}
